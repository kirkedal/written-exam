% !TEX root = master.tex

% Questions and answers by Troels Henriksen

\questionhead{Operating Systems}{80 minutes}
\lstset{language=C}

\subquestionhead{True/False Questions}{8 minutes}

\begin{tfquestion}
\tfitem[true]{\lstinline{fopen()} is not a system call.}
\tfitem[false]{There is never more physical memory than virtual memory.}
\tfitem[false]{System calls are implemented via signals.}
\tfitem[false]{Virtual memory requires a disk.}
\tfitem[false]{System calls run in user mode.}
\tfitem[true]{Condition variables cannot be efficiently implemented solely with mutexes.}
\end{tfquestion}

\subquestionhead{Multiple Choice Questions}{12 minutes}
\emph{In each of the following questions, you may put one or more answers.}

\begin{mcquestion}
Which of the following operations are guaranteed to execute atomically?

\mcitem{\texttt{pthread\_cond\_signal()}}
\mcitem[check]{\texttt{pthread\_mutex\_lock()}}
\mcitem{\texttt{x++} (when \texttt{x} is \texttt{int})}
\mcitem{\texttt{memcpy(\&x, \&y, sizeof(x))}}
\mcitem[check]{\texttt{pthread\_cond\_wait()}}
\mcitem{\texttt{exit(0)}}
\end{mcquestion}

\begin{mcquestion}
Consider a demand-paged system with the following time-measured utilisations:

  \begin{tabular}{lr}
    CPU utilisation & 50\% \\
    Paging disk & 0.7\% \\
    Other I/O devices & 75\%
  \end{tabular}

\noindent Which of the following would likely improve CPU utilisation?

\mcitem[check]{Install a faster CPU.}
\mcitem{Install a bigger paging disk.}
\mcitem{Install a faster paging disk.}
\mcitem{Install more main memory.}
\mcitem[check]{Increase the degree of multiprogramming.}
\end{mcquestion}

\newpage
\subquestionhead{Long Questions}{36 minutes}

\begin{question}
  Which of the following programming techniques and data structures
  are ``good'' for a demand-paged environment, and which are ``bad''
  (performance-wise)?  Explain your answers.

  \begin{itemize}
  \item Stack
  \item Hash table
  \item Sequential search of array
  \item Sequential search of linked list
  \item Binary search of array
  \item Vector operations (such as vector addition or computing dot products)
  \end{itemize}

  \answerlinesfill[A stack is efficient because operations exhibit
  good \textit{locality}---we are always operating on addresses near
  each other, which minimises page faults.  A hash table can be
  inefficient, because a good hash algorithm will ensure that accesses
  are evenly distributed among buckets.  A sequential search of an
  array is efficient, because of good locality.  Sequentially
  searching of a linked list is likely inefficient, because logically
  neighboring nodes can be arbitrarily distant in memory.  Binary
  search of an array is likewise also inefficient (from a memory
  access point of view), again because of large jumps in addresses.
  Vector operations tend to be efficient, because they involve
  sequentially traversing arrays.]
\end{question}

% \answerfigure
